\documentclass{article}
\usepackage[utf8]{inputenc}
\usepackage{float}

\title{Comparison of Containerized and Native Applications in the Store}
\author{Zac Freeman}

\begin{document}

\maketitle

\section{Introduction}
This paper aims to summarize the impact of Docker on the store systems and processes. The current impact of Docker will be discussed and analyzed in Deployment Process, Resource Consumption, and Latency. The future impact of Docker will be discussed in Development Experience and Application Security. Any code, data, or assumptions used in the analyses will be listed in Methods, Data, and Scripts.

\section{Deployment Process}
Currently, the Z-neXt project employs Docker to manage a containerized Envoy application. Envoy routes requests from in-store applications to either an above-store or in-store service, depending on the health of the above-store service.

\subsection{Container Deployment}
Deploying the Envoy container to a store requires that the Docker daemon be installed in the store and the official Envoy Docker image be downloaded from Docker Hub. Installing the Docker daemon necessitates enabling the ol7\_addons-prod package repository in the store. Once the image is downloaded to the store, it can be started by the Docker daemon.

This solution could be made production-ready by hosting the Docker image in the AutoZone Artifactory, and retrieving it in the store from the AutoZone Artifactory.

\subsection{Native Deployment}
Deploying the native Envoy application to a store would require adding the tetrate-getenvoy package repository to the store. Once the package repository is added, the Envoy application can be installed with the getenvoy-envoy package.

\section{Resource Consumption}
The foundation of the Docker container is the Linux namespace. A Linux namespace encapsulates a process to limit its access to system resources. A Docker container is able to make use of the running Linux kernel and additionally provide the dependencies needed by the containerized software in a virtual environment. This approach provides a lightweight solution to software virtualization.

The resource consumption of a containerized application is broken up into two groups. The first group, labelled "Docker Dependencies", consists of the per-system applications such as the docker daemon, the container daemon, etc., that represent a static cost that does not increase with the number of containers. The other group, labelled "Containerized Application", consists of the container and the application itself, which gives an approximation of the per-application cost of a containerized application.

\subsection{Idle Consumption}
The following data was collected while no requests were being made to the Envoy application.

\begin{table}[H]
\begin{tabular}{ |c|c|c|c| }
 \hline
   & Memory (MB) & Storage (MB) & CPU (\%) \\ 
 \hline
 Native Application & 36.3 & 93.9 & 0.47 \\
 \hline
 Containerized Application & 79.8 & 84.6 & 0.55 \\
 \hline
 Docker Dependencies & 93.2 & 359.4 & 0.09 \\
 \hline\hline
 Increase (with Dependencies) & 136.7 & 350.1 & 0.17 \\
 \hline
 Increase (without Dependencies) & 43.5 & -9.3 & 0.08 \\
 \hline
\end{tabular}
\caption{The store resources consumed by the native application and the containerized application when idle.}
\label{table:1}
\end{table}

The relative increase in Memory, Storage, and CPU consumption between the native application and the containerized application, without dependencies, are 120\%, -10\%, and 17\%, respectively. Similar increases can be anticipated for each additional containerized application in the store.

\subsection{Consumption Under Load}
The following data was collected while the Envoy application was under a simulated load of 10 requests per second.

\begin{table}[H]
\begin{tabular}{ |c|c|c| }
 \hline
   & Memory (MB) & CPU (\%) \\ 
 \hline
 Native Application & 38.5 & 3.75 \\
 \hline
 Containerized Application & 85.9 & 3.91 \\
 \hline
 Docker Dependencies & 97.2 & 0.05 \\
 \hline\hline
 Increase (with Dependencies) & 144.6 & 0.21 \\
 \hline
 Increase (without Dependencies) & 47.4 & 0.16 \\
 \hline
\end{tabular}
\caption{The store resources consumed by the native application and the containerized application under load.}
\label{table:2}
\end{table}

The relative increase in Memory and CPU consumption between the native application and the containerized application, without dependencies, are 123\% and 4\%, respectively. Similar increases can be anticipated for each additional containerized application in the store.

\subsection{Interpretation}
Across the board, the containerized Envoy application requires more resources to run than the native Envoy application. Perhaps a more significant result is how the costs associated with Docker scale with the number of containers and the load applied to the application.

The Docker Dependencies' usage of memory, storage, and CPU time does not scale with the number of containers or the load on the containerized application.

The container that envelops the containerized application does not scale its resource consumption with the load on the containerized application. Containerizing an application represents a flat resource cost, with respect to load.

\section{Latency}

\section{Development Experience}

\section{Application Security}

\section{Methods}
\subsection{Storage Consumption}
The size of each application was captured with \texttt{rpm -qi package\_name}. The size of the docker image used by the containerized application was captured from \texttt{docker images}.

\subsection{Memory Consumption}
The memory consumed by each process was equated with the physical memory being used by each process. The physical memory being used by each process was captured with \texttt{cat /proc/process\_id/status $|$ grep VmRSS}.

\subsection{CPU Consumption}
CPU consumption for each process was determined by taking the average of 100 values collected from \texttt{top} over a period of 200 seconds. \texttt{cpu-usage.sh} is used to collect these values and \texttt{average.pl} is used to average the values.

\subsection{Load Testing}
To simulate a heavy network load on the envoy applications, a \texttt{load-test.sh} was run to make 10 requests per second to \texttt{https://localhost:10443/v1/items/en-us}. Envoy was configured to route these requests to a local service, to avoid stressing an above store application.

\section{Data}
\subsection{Storage Consumption}
The storage consumption listed for the Docker Dependencies represents the storage consumed by the packages added to use Docker in the store. This includes \texttt{docker-engine}, \texttt{docker-cli}, \texttt{containerd}, and \texttt{container-selinux}.

\begin{table}[H]
\begin{tabular}{ |c|c| }
 \hline
   & Storage (MB) \\ 
 \hline
 \texttt{docker-engine} & 103.7 \\
 \hline
 \texttt{docker-cli} & 168.7 \\
 \hline
 \texttt{containerd} & 87.0 \\
 \hline
 \texttt{container-selinux} & $>$0.1 \\
 \hline\hline
 Total & 359.4 \\
 \hline
\end{tabular}
\caption{A breakdown of the storage space consumed by each package that contributes to total storage consumption listed for the Docker Dependencies.}
\label{table:3}
\end{table}

The storage consumption listed for the Containerized Application is just the storage occupied by the Docker image used by the Envoy container.

\subsection{Memory and CPU Consumption}
The Memory and CPU consumption listed for the Docker Dependencies represents the memory and CPU time consumed by the applications required to support Docker containers in the store. This includes the \texttt{dockerd}, \texttt{containerd}, and \texttt{containerd-shim}.

The Memory and CPU consumption listed for the Containerized Application represents the memory and CPU time consumed by the Envoy application and the container that envelops it.

\begin{table}[H]
\begin{tabular}{ |c|c|c| }
 \hline
   & Memory (MB) & CPU (\%) \\ 
 \hline
 \texttt{dockerd} & 75.1 & 0.05 \\
 \hline
 \texttt{containerd} & 10.8 & 0.03 \\
 \hline
 \texttt{containerd-shim} & 7.3 & 0.01 \\
 \hline\hline
 Total & 93.2 & 0.09 \\
 \hline
\end{tabular}
\caption{A breakdown of the memory and CPU time consumed by the applications underlying Docker while the containerized application is idle.}
\label{table:4}
\end{table}

\begin{table}[H]
\begin{tabular}{ |c|c|c| }
 \hline
   & Memory (MB) & CPU (\%) \\ 
 \hline
 \texttt{dockerd} & 75.1 & 0.02 \\
 \hline
 \texttt{containerd} & 14.8 & 0.03 \\
 \hline
 \texttt{containerd-shim} & 7.3 & $>$0.01 \\
 \hline\hline
 Total & 97.2 & 0.05 \\
 \hline
\end{tabular}
\caption{A breakdown of the memory and CPU time consumed by the applications underlying Docker while the containerized application is under load.}
\label{table:5}
\end{table}

\begin{table}[H]
\begin{tabular}{ |c|c|c| }
 \hline
   & Memory (MB) & CPU (\%) \\ 
 \hline
 \texttt{Envoy} & 32.3 & 0.54 \\
 \hline
 \texttt{Docker Container} & 47.5 & 0.01 \\
 \hline\hline
 Total & 79.8 & 0.05 \\
 \hline
\end{tabular}
\caption{A breakdown of the memory and CPU time consumed by the envoy application and its container while envoy is idle.}
\label{table:6}
\end{table}

\begin{table}[H]
\begin{tabular}{ |c|c|c| }
 \hline
   & Memory (MB) & CPU (\%) \\ 
 \hline
 \texttt{Envoy} & 38.9 & 3.90 \\
 \hline
 \texttt{Docker Container} & 47.0 & 0.01 \\
 \hline\hline
 Total & 85.9 & 3.91 \\
 \hline
\end{tabular}
\caption{A breakdown of the memory and CPU time consumed by the envoy application and its container while envoy is under load.}
\label{table:7}
\end{table}

\section{Scripts}
\subsection{average.pl}
\begin{verbatim}
#!/usr/bin/perl

use strict;

my $total = 0;
my $count = 0;
while (my $line = <>) {
      $total += $line;
      $count++;
}
print "Average = ", $total / $count, "\n";
\end{verbatim}

\subsection{cpu-usage.sh}
\begin{verbatim}
#!/bin/bash

if [[ $# -lt 2 ]]
then
    echo "missing arguments"
    exit
fi

pid="$1"
output="$2"
interval=1
num=100
if [[ $# -ge 3 ]]
then
    interval="$3"
fi

if [[ $# -eq 4 ]]
then
    num="$4"
fi

echo "recording CPU usage of process $pid $num times every 2*$interval seconds into $output..."

rm "$output"

for (( count=0; count<num; count++ ))
do
    top -b -n 2 -d "$interval" -p "$pid" | tail -1 | awk '{print $9}' >> "$output"
done
\end{verbatim}

\subsection{load-test.sh}
\begin{verbatim}
#!/bin/bash

if [[ $# -lt 1 ]]
then
    echo "missing url argument"
    exit
fi

url="$1"
interval=0.1
if [[ $# -eq 2 ]]
then
  interval="$2"
fi

echo "making requests to $url every $interval seconds..."

while true
do
    curl --insecure --silent --output "/dev/null" "$url" &
    sleep "$interval"
done
\end{verbatim}

\end{document}
